% GAME ON!
% The Magic Comment
% !TeX program = XeLaTeX
\documentclass[11pt,a4paper,oneside]{article}
\usepackage[colorlinks=true,urlcolor=blue,citecolor=black,hyperfootnotes=false,%
linkcolor=black]{hyperref}
\usepackage{xltxtra,amsmath,amssymb,latexsym,esint}
\usepackage{siunitx}
%For Chinese
\usepackage[UTF8,heading=true]{ctex}
\ctexset{autoindent=true}
%For bibliography's style, according to the latest national standard
\usepackage[backend=biber,style=gb7714-2015,gbpub=true]{biblatex}
%To use biblatex, first
%$ xelatex thesis.tex
%then
%$ biber thesis
%and last
%$ xelatex thesis.tex
%Bibdata import:
\addbibresource{bylw.bib}
%customize the heading of the reference as the de facto "standard", 
%delete this block will cause the heading of the bibliography return to normal
%the block begin
\defbibheading{mybibintoc}[\bibname]{\hspace{-1.8pc}\songti\textbf{#1:}%
\addcontentsline{toc}{section}{\hspace{-1.8pc}#1}}%\hspace{-1.8pc} in order to line up in toc
\defbibenvironment{mypubs}%define new bibenv for shrinking the size of the font in each entry
 {\list{}
    {\setlength{\leftmargin}{\bibhang}%
    \setlength{\itemindent}{-\leftmargin}%
     \setlength{\itemsep}{\bibitemsep}%
     \setlength{\parsep}{\bibparsep}}}
  {\endlist}
  {\item\small%notice that we shrink our font here
   \printtext[labelnumberwidth]{%
    \printfield{labelprefix}%
    \printfield{labelnumber}}%
    \addspace%
  }
%the block end
%Set the global margin and layout as mandated
\usepackage[top=2.54cm,bottom=2.54cm,left=4cm,right=3.17cm]{geometry}
%set the page number to its proper place
\usepackage{fancyhdr}
\pagestyle{fancy}
\fancyhf{}
\fancyhead{}%Blank out the default header
\renewcommand{\headrulewidth}{0pt}%as mandated, remove the headrule
%The de facto "Standard" does not specify the necessity of footrulewidth
\fancyfoot{}%Blank out the default footer
\fancyfoot[C]{\thepage}%as mandated
%Initializing the Title segments

%Warning: Never using footnote 
%If you delete this block, set the hyperfootnotes=true
%Footnotestart
\usepackage{tikz}
\newcommand*\circled[1]{%
  \tikz[baseline=(char.base)]\node[shape=circle,draw,inner sep=0.1pt,%
  font=\tiny,minimum size=7pt] (char) {#1};}
\renewcommand\thefootnote{\protect\circled{\arabic{footnote}}}
%Footnoteend
%Now we come to the table of contents, alash! What a Chaos!
\usepackage{xcolor}
%%\usepackage{tocloft}
%Now we customize the entries' lines
%%\renewcommand{\cftdot}{\ensuremath{\cdot}}
%Now we set the leftmost words in the table of contents to be black
%%\renewcommand\cftsecfont{\color{black}\bfseries}
%%\renewcommand\cftsecpagefont{\color{black}\bfseries}
%%\renewcommand\cftsubsecfont{\bfseries}
%%\renewcommand\cftsubsecpagefont{\bfseries}
%And fill in the page number on the right and the corresponding names of sections on
%the left with dots
%The following command will set the dotslines of each entry
%the number inducates the density of the dots
%%\renewcommand\cftsecdotsep{1}
%%\renewcommand\cftsubsecdotsep{1}
%We also set the level of the commands
% tocdepth value determines to which level the sectioning commands are printed in the ToC
\setcounter{tocdepth}{4}
%secnumdepth value determines up to what level the sectioning titles are numbered.
%They are LaTeX counters and you can set them using \setcounter. The number starts from -1.
\setcounter{secnumdepth}{4}
\usepackage{titletoc}
%\newcommand{\mydots}{\xleaders\hbox{$\cdot$}\hfill\kern0pt}
%we define a new command cuz we want to fill in a line with specific symbols, \xleaders does the job
\newcommand{\mydots}[1][$\cdot$]{\xleaders\hbox{#1}\hfill\kern0pt}
\titlecontents{section}[1.8pc]
  {\addvspace{3pt}\bfseries}%we here customize our leftmost entries in toc
  {\contentslabel[\large\thecontentslabel.]{1.8pc}}%in the squre bracket[ here] ,the format is sectionnumber.
  %1.8pc simply shift the entries in the toc to the right by 1.8pc
  {}
  {\hspace{1ex}\mydots\hspace{1ex}\thecontentspage}%

\titlecontents*{subsection}[1.8pc]
  {\bfseries}
  {\thecontentslabel. }
  {}
  { \hspace{1ex}\mydots\hspace{1ex}\thecontentspage}%
%We override the \tableofcontents command to customize our own
\makeatletter
\renewcommand{\tableofcontents}{%
{\hfill\songti\bfseries\huge 目\phantom{空空}录\phantom{望美}\hfill}%
\vspace{1cm}
\@starttoc{toc}}
\makeatother
%
%
%Frankly, I want to F*CK the fella who write the code of CTEX
%The following code shall restore the section title to the left of the
%document where it shall be resided.
\makeatletter
\g@addto@macro{\CTEX@section@format}{\raggedright}
\makeatother
%\ctexset{%
%      section/format=\Large\bfseries\heiti%
%      subsection/format=\large\bfseries\heiti%
%}
%We define the specific command to do conlusion job in a Chinese Style
\newcommand{\zhconclusion}{\section*{\heiti\large 结束语}%
\addcontentsline{toc}{section}{\hspace{-1.8pc}\heiti 结束语}}%\hspace{-1.8pc} in order to line up in toc
\newcommand{\chconclusion}{}
\newenvironment{chineseconclusion}{\begin{chconclusion}\zhconclusion}%
{\vspace{1ex}\\ \end{chconclusion}}
%
%revise the section command
\newcommand{\csection}[1]{\section{\heiti\large #1}}
\newcommand{\csubsection}[1]{\subsection{\fangsong #1}}
%Define SanHao fontsize
\newcommand{\sanhao}{\fontsize{16pt}{16pt}\selectfont}
%De facto standard doesn't endorse outline heading
\renewcommand{\abstractname}{}
%---------------------------------------------------------------------------------------------------
%Your thesis title here
\newcommand{\ahnutitle}{%
%write here your thesis title
论文题目
}
%Your name, here
\newcommand{\ahnuauthor}{%
%write here your thesis title
作者
}
\newcommand{\ahnuab}{%
%write your abstract in Chinese here
店内可三年级怎么就算哦刺死第三就算了看错了亏损空间的脸色就比赛开始大家女性所看见的小鸟送所能想到第三开门就相似。
}
\newcommand{\ahnukeys}{%
%Keywords in Chinese
关键;关键;挂你结案
}
%=============================================================================================================
\newcommand{\ahnuetitle}{%
%write here your thesis title
Title of Thesis
}
%Your name, here
\newcommand{\ahnueauthor}{%
%write here your thesis title
Author
}
\newcommand{\ahnueab}{%
%write your abstract in English here
Rykener said that he was introduced to sexual contact with men by Elizabeth Brouderer, a London [[embroideress]] who dressed him as a woman and may have acted as his procurer. According to his account, he had sex with both men and women, including priests and nuns. Rykener spent part of summer 1394 in [[Oxford]], working both as a prostitute and as an embroideress. He later mentioned that in [[Beaconsfield]] he had a sexual relationship with a woman. Rykener returned to London via [[Burford]] in Oxfordshire, where he worked as a barmaid and continued with sex work. On his return to London, he had paid encounters near the [[Tower of London]], just outside the City. Rykener was arrested with Britby on the evening of Sunday, 11 December. Rykener was in women's clothes, which he was still wearing during his subsequent interrogation. It was there that he described his encounters—and his sexual history—in great detail. It appears that no charges were ever brought against him; or at least, no records have been found suggesting so. Nothing definite is known of Rykener after his interrogation; he has been tentatively identified as a John Rykener imprisoned by and escaping from the [[Bishop of London]] in 1399.

}
\newcommand{\ahnuekeys}{%
%Keywords in English
key;key;key
}
%-------------------------------------------------------------------------------------------------------------
%Reformulate the format of label
\usepackage{titlesec}
\titlelabel{\thetitle.\quad}
\titleformat*{\subsection}{\large\fangsong}
%Now let's go
\begin{document}
%The TOC
\thispagestyle{empty}
\tableofcontents
\newpage
%The thesis begin
\setcounter{page}{1}
\begin{center}
{\heiti\sanhao\ahnutitle }\\
\vspace{12pt}
\ahnuauthor , 物理与电子信息学院
\end{center}
\vspace{-1.2cm}
\begin{abstract}
  \noindent{\large\heiti 摘要:}{\fangsong\ahnuab}

  \noindent{\large\heiti 关键词:}{\fangsong\ahnukeys}
\end{abstract}
\vspace{0pt}
\begin{center}
  {\bfseries\sanhao\ahnuetitle }\\
  \vspace{12pt}
  \ahnueauthor,  College of Physics and ELectronic Information
  \end{center}
  \vspace{-1.2cm}
  \begin{abstract}
    \noindent{\large\bfseries Abstract: }{\ahnueab}
  
    \noindent{\large\bfseries Keywords: }{\ahnuekeys}
  \end{abstract}
%\\\\\\\\\\\\\\\\\\\\\\\\\\\\\\\\\\\\\\\\\\\\\\\\\\\\\\\\\\\\\\\\\\\\\\\\\\\\\\\\\\\\\\\\\\\\\\\\\\\\\\\\\\\\\\\\\
%Start your thesis here
\csection{第一节}
dnhi hewiwo ef\fangsong 仿宋

Rykener said that he was introduced to sexual contact with men by Elizabeth Brouderer, a London [[embroideress]] who dressed him as a woman and may have acted as his procurer. According to his account, he had sex with both men and women, including priests and nuns. Rykener spent part of summer 1394 in [[Oxford]], working both as a prostitute and as an embroideress. He later mentioned that in [[Beaconsfield]] he had a sexual relationship with a woman. Rykener returned to London via [[Burford]] in Oxfordshire, where he worked as a barmaid and continued with sex work. On his return to London, he had paid encounters near the [[Tower of London]], just outside the City. Rykener was arrested with Britby on the evening of Sunday, 11 December. Rykener was in women's clothes, which he was still wearing during his subsequent interrogation. It was there that he described his encounters—and his sexual history—in great detail. It appears that no charges were ever brought against him; or at least, no records have been found suggesting so. Nothing definite is known of Rykener after his interrogation; he has been tentatively identified as a John Rykener imprisoned by and escaping from the [[Bishop of London]] in 1399.

Historians of social, sexual and gender history are especially interested in Rykener's case because of what it reveals about medieval views on sex and gender. [[Jeremy Goldberg]], for example, views it firmly in the context of [[King Richard II]]'s quarrel with the city of London—although he has also questioned the veracity of the entire record, and posited that the case was merely a propaganda piece by city officials. Historian James A. Schultz has viewed the affair as being of greater significance to historians than more famous medieval stories such as [[Tristan and Iseult]]. [[Ruth Mazo Karras]]—who in the 1990s rediscovered the Rykener case in the [[London Metropolitan Archives|City of London archives]]—sees it as illustrating the difficulties the law has in addressing things it cannot describe. Modern interest in John/Eleanor Rykener has not been confined to [[academia]]. Rykener has appeared as a character in at least one work of popular [[historical fiction]], and his story has been adapted for the stage.

\csubsection{第一小节}
hgdexhbudhnxiohw
\[
\varoiint_\mathcal{S} \nabla\cdot 
\]
\csection{第二节}
gxybedgwi

c\cite{hall1968effect}
\newpage
Rykener said that he was introduced to sexual contact with men by Elizabeth Brouderer, a London [[embroideress]] who dressed him as a woman and may have acted as his procurer. According to his account, he had sex with both men and women, including priests and nuns. Rykener spent part of summer 1394 in [[Oxford]], working both as a prostitute and as an embroideress. He later mentioned that in [[Beaconsfield]] he had a sexual relationship with a woman. Rykener returned to London via [[Burford]] in Oxfordshire, where he worked as a barmaid and continued with sex work. On his return to London, he had paid encounters near the [[Tower of London]], just outside the City. Rykener was arrested with Britby on the evening of Sunday, 11 December. Rykener was in women's clothes, which he was still wearing during his subsequent interrogation. It was there that he described his encounters—and his sexual history—in great detail. It appears that no charges were ever brought against him; or at least, no records have been found suggesting so. Nothing definite is known of Rykener after his interrogation; he has been tentatively identified as a John Rykener imprisoned by and escaping from the [[Bishop of London]] in 1399.
\newpage
d\footnote{中华民国}

d\footnote{hua}
\newpage
\begin{chineseconclusion}
  Rykener said that he was introduced to sexual contact with men by Elizabeth Brouderer, a London [[embroideress]] who dressed him as a woman and may have acted as his procurer. According to his account, he had sex with both men and women, including priests and nuns. Rykener spent part of summer 1394 in [[Oxford]], working both as a prostitute and as an embroideress. He later mentioned that in [[Beaconsfield]] he had a sexual relationship with a woman. Rykener returned to London via [[Burford]] in Oxfordshire, where he worked as a barmaid and continued with sex work. On his return to London, he had paid encounters near the [[Tower of London]], just outside the City. Rykener was arrested with Britby on the evening of Sunday, 11 December. Rykener was in women's clothes, which he was still wearing during his subsequent interrogation. It was there that he described his encounters—and his sexual history—in great detail. It appears that no charges were ever brought against him; or at least, no records have been found suggesting so. Nothing definite is known of Rykener after his interrogation; he has been tentatively identified as a John Rykener imprisoned by and escaping from the [[Bishop of London]] in 1399.

  Historians of social, sexual and gender history are especially interested in Rykener's case because of what it reveals about medieval views on sex and gender. [[Jeremy Goldberg]], for example, views it firmly in the context of [[King Richard II]]'s quarrel with the city of London—although he has also questioned the veracity of the entire record, and posited that the case was merely a propaganda piece by city officials. Historian James A. Schultz has viewed the affair as being of greater significance to historians than more famous medieval stories such as [[Tristan and Iseult]]. [[Ruth Mazo Karras]]—who in the 1990s rediscovered the Rykener case in the [[London Metropolitan Archives|City of London archives]]—sees it as illustrating the difficulties the law has in addressing things it cannot describe. Modern interest in John/Eleanor Rykener has not been confined to [[academia]]. Rykener has appeared as a character in at least one work of popular [[historical fiction]], and his story has been adapted for the stage.  
\end{chineseconclusion}

\phantom{empty}

\printbibliography[heading=mybibintoc,env=mypubs,title=参考文献]
\end{document}
%Game over!